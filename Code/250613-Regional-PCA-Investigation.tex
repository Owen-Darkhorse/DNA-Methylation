% Options for packages loaded elsewhere
\PassOptionsToPackage{unicode}{hyperref}
\PassOptionsToPackage{hyphens}{url}
%
\documentclass[
]{article}
\usepackage{amsmath,amssymb}
\usepackage{iftex}
\ifPDFTeX
  \usepackage[T1]{fontenc}
  \usepackage[utf8]{inputenc}
  \usepackage{textcomp} % provide euro and other symbols
\else % if luatex or xetex
  \usepackage{unicode-math} % this also loads fontspec
  \defaultfontfeatures{Scale=MatchLowercase}
  \defaultfontfeatures[\rmfamily]{Ligatures=TeX,Scale=1}
\fi
\usepackage{lmodern}
\ifPDFTeX\else
  % xetex/luatex font selection
\fi
% Use upquote if available, for straight quotes in verbatim environments
\IfFileExists{upquote.sty}{\usepackage{upquote}}{}
\IfFileExists{microtype.sty}{% use microtype if available
  \usepackage[]{microtype}
  \UseMicrotypeSet[protrusion]{basicmath} % disable protrusion for tt fonts
}{}
\makeatletter
\@ifundefined{KOMAClassName}{% if non-KOMA class
  \IfFileExists{parskip.sty}{%
    \usepackage{parskip}
  }{% else
    \setlength{\parindent}{0pt}
    \setlength{\parskip}{6pt plus 2pt minus 1pt}}
}{% if KOMA class
  \KOMAoptions{parskip=half}}
\makeatother
\usepackage{xcolor}
\usepackage[margin=1in]{geometry}
\usepackage{color}
\usepackage{fancyvrb}
\newcommand{\VerbBar}{|}
\newcommand{\VERB}{\Verb[commandchars=\\\{\}]}
\DefineVerbatimEnvironment{Highlighting}{Verbatim}{commandchars=\\\{\}}
% Add ',fontsize=\small' for more characters per line
\usepackage{framed}
\definecolor{shadecolor}{RGB}{248,248,248}
\newenvironment{Shaded}{\begin{snugshade}}{\end{snugshade}}
\newcommand{\AlertTok}[1]{\textcolor[rgb]{0.94,0.16,0.16}{#1}}
\newcommand{\AnnotationTok}[1]{\textcolor[rgb]{0.56,0.35,0.01}{\textbf{\textit{#1}}}}
\newcommand{\AttributeTok}[1]{\textcolor[rgb]{0.13,0.29,0.53}{#1}}
\newcommand{\BaseNTok}[1]{\textcolor[rgb]{0.00,0.00,0.81}{#1}}
\newcommand{\BuiltInTok}[1]{#1}
\newcommand{\CharTok}[1]{\textcolor[rgb]{0.31,0.60,0.02}{#1}}
\newcommand{\CommentTok}[1]{\textcolor[rgb]{0.56,0.35,0.01}{\textit{#1}}}
\newcommand{\CommentVarTok}[1]{\textcolor[rgb]{0.56,0.35,0.01}{\textbf{\textit{#1}}}}
\newcommand{\ConstantTok}[1]{\textcolor[rgb]{0.56,0.35,0.01}{#1}}
\newcommand{\ControlFlowTok}[1]{\textcolor[rgb]{0.13,0.29,0.53}{\textbf{#1}}}
\newcommand{\DataTypeTok}[1]{\textcolor[rgb]{0.13,0.29,0.53}{#1}}
\newcommand{\DecValTok}[1]{\textcolor[rgb]{0.00,0.00,0.81}{#1}}
\newcommand{\DocumentationTok}[1]{\textcolor[rgb]{0.56,0.35,0.01}{\textbf{\textit{#1}}}}
\newcommand{\ErrorTok}[1]{\textcolor[rgb]{0.64,0.00,0.00}{\textbf{#1}}}
\newcommand{\ExtensionTok}[1]{#1}
\newcommand{\FloatTok}[1]{\textcolor[rgb]{0.00,0.00,0.81}{#1}}
\newcommand{\FunctionTok}[1]{\textcolor[rgb]{0.13,0.29,0.53}{\textbf{#1}}}
\newcommand{\ImportTok}[1]{#1}
\newcommand{\InformationTok}[1]{\textcolor[rgb]{0.56,0.35,0.01}{\textbf{\textit{#1}}}}
\newcommand{\KeywordTok}[1]{\textcolor[rgb]{0.13,0.29,0.53}{\textbf{#1}}}
\newcommand{\NormalTok}[1]{#1}
\newcommand{\OperatorTok}[1]{\textcolor[rgb]{0.81,0.36,0.00}{\textbf{#1}}}
\newcommand{\OtherTok}[1]{\textcolor[rgb]{0.56,0.35,0.01}{#1}}
\newcommand{\PreprocessorTok}[1]{\textcolor[rgb]{0.56,0.35,0.01}{\textit{#1}}}
\newcommand{\RegionMarkerTok}[1]{#1}
\newcommand{\SpecialCharTok}[1]{\textcolor[rgb]{0.81,0.36,0.00}{\textbf{#1}}}
\newcommand{\SpecialStringTok}[1]{\textcolor[rgb]{0.31,0.60,0.02}{#1}}
\newcommand{\StringTok}[1]{\textcolor[rgb]{0.31,0.60,0.02}{#1}}
\newcommand{\VariableTok}[1]{\textcolor[rgb]{0.00,0.00,0.00}{#1}}
\newcommand{\VerbatimStringTok}[1]{\textcolor[rgb]{0.31,0.60,0.02}{#1}}
\newcommand{\WarningTok}[1]{\textcolor[rgb]{0.56,0.35,0.01}{\textbf{\textit{#1}}}}
\usepackage{graphicx}
\makeatletter
\newsavebox\pandoc@box
\newcommand*\pandocbounded[1]{% scales image to fit in text height/width
  \sbox\pandoc@box{#1}%
  \Gscale@div\@tempa{\textheight}{\dimexpr\ht\pandoc@box+\dp\pandoc@box\relax}%
  \Gscale@div\@tempb{\linewidth}{\wd\pandoc@box}%
  \ifdim\@tempb\p@<\@tempa\p@\let\@tempa\@tempb\fi% select the smaller of both
  \ifdim\@tempa\p@<\p@\scalebox{\@tempa}{\usebox\pandoc@box}%
  \else\usebox{\pandoc@box}%
  \fi%
}
% Set default figure placement to htbp
\def\fps@figure{htbp}
\makeatother
\setlength{\emergencystretch}{3em} % prevent overfull lines
\providecommand{\tightlist}{%
  \setlength{\itemsep}{0pt}\setlength{\parskip}{0pt}}
\setcounter{secnumdepth}{-\maxdimen} % remove section numbering
\usepackage{bookmark}
\IfFileExists{xurl.sty}{\usepackage{xurl}}{} % add URL line breaks if available
\urlstyle{same}
\hypersetup{
  pdftitle={240614-},
  pdfauthor={Jingxin Wang},
  hidelinks,
  pdfcreator={LaTeX via pandoc}}

\title{240614-}
\author{Jingxin Wang}
\date{2025-06-13}

\begin{document}
\maketitle

\subsection{To Dos}\label{to-dos}

\begin{enumerate}
\def\labelenumi{\arabic{enumi}.}
\tightlist
\item
  Apply INT and log transformation to methylation data and apply
  classical PCA. Contrast these results with classical PCA on the raw
  data
\item
  Read the regional PCA paper in details, identify what they do with the
  score matrix for each gene region, and identify the research question
  they are trying to answer.
\item
  Read over the robust PCA and understand the mathematical principles
\end{enumerate}

\subsection{Important Update}\label{important-update}

\begin{enumerate}
\def\labelenumi{\arabic{enumi}.}
\tightlist
\item
  The project will be divided into 4 parts, and we are currently at
  stage 1 with a brief touch in stage 2.
\end{enumerate}

\begin{itemize}
\tightlist
\item
  Preprocess the data
\item
  Filtering out important genes (for each cancer types)
\item
  Build gene networks
\item
  Compare different networks
\end{itemize}

\begin{enumerate}
\def\labelenumi{\arabic{enumi}.}
\setcounter{enumi}{1}
\tightlist
\item
  Regional PCA relies on pre-defined segmentation of genes, but we want
  to come up with natural groupings of genes. This will be innovation
  target we try to achieve.
\end{enumerate}

\section{1. What is the goal of the
research?}\label{what-is-the-goal-of-the-research}

Identify cell-type specific DNA methylation changes that are associated
with AD phenotypes.

\section{2. What did they do with the score
matrix?}\label{what-did-they-do-with-the-score-matrix}

\begin{enumerate}
\def\labelenumi{\arabic{enumi}.}
\tightlist
\item
  Simulation study: to show that rPCA has a greater statistical power
  than simple averaging in finding the DMRs for every proportion of CpG
  sites and methlylation differences.
\item
  Real dataset: the PCA score matrix \(Z_r = X V_r\) is calculated for
  each gene region, and each column of \(Z_r\) is treated as a rPC.
  Next, a \texttt{lmFit} is performed on each score matrix \(Z-r\) to
  identify DMRs.
\item
  \texttt{lmFit} fits multiple linear models by weighted or generalized
  least squares.
\end{enumerate}

\section{Test Log Transformation and IVT as Data Preprocessing
Step}\label{test-log-transformation-and-ivt-as-data-preprocessing-step}

\begin{Shaded}
\begin{Highlighting}[]
\DocumentationTok{\#\# Function Definition}
\FunctionTok{source}\NormalTok{(}\StringTok{"IVT.R"}\NormalTok{)}
\FunctionTok{source}\NormalTok{(}\StringTok{"pcaCombo.R"}\NormalTok{)}

\DocumentationTok{\#\# Load Data}
\FunctionTok{library}\NormalTok{(tidyverse)}
\end{Highlighting}
\end{Shaded}

\begin{verbatim}
## Warning: package 'ggplot2' was built under R version 4.4.3
\end{verbatim}

\begin{verbatim}
## Warning: package 'purrr' was built under R version 4.4.3
\end{verbatim}

\begin{Shaded}
\begin{Highlighting}[]
\NormalTok{X.T }\OtherTok{\textless{}{-}}  \FunctionTok{readRDS}\NormalTok{(}\AttributeTok{file =} \StringTok{"../Data/Common\_pan\_cancer\_hyper\_bins\_adjusted\_and\_normalized\_cnt\_in\_SE.RDS"}\NormalTok{)}
\FunctionTok{dim}\NormalTok{(X.T)}
\end{Highlighting}
\end{Shaded}

\begin{verbatim}
## [1] 24418   378
\end{verbatim}

\begin{Shaded}
\begin{Highlighting}[]
\NormalTok{sampleInfo }\OtherTok{\textless{}{-}} \FunctionTok{read.csv}\NormalTok{(}\StringTok{"../Data/Common\_pan\_cancer\_hyper\_bins\_adjusted\_cnt\_in\_SE\_samples\_info.csv"}\NormalTok{)}
\NormalTok{cancerTypes }\OtherTok{\textless{}{-}}\NormalTok{ sampleInfo}\SpecialCharTok{$}\NormalTok{cancer\_type[sampleInfo}\SpecialCharTok{$}\NormalTok{sample\_id }\SpecialCharTok{\%in\%} \FunctionTok{colnames}\NormalTok{(X.T)]}

\DocumentationTok{\#\# Raw data without preprocessing}
\NormalTok{PC\_raw }\OtherTok{=} \FunctionTok{pcaCombo}\NormalTok{(}\FunctionTok{t}\NormalTok{(X.T))}
\end{Highlighting}
\end{Shaded}

\pandocbounded{\includegraphics[keepaspectratio]{250613-Regional-PCA-Investigation_files/figure-latex/unnamed-chunk-1-1.pdf}}

\begin{Shaded}
\begin{Highlighting}[]
\DocumentationTok{\#\# Log Transformation}
\NormalTok{logX.T }\OtherTok{\textless{}{-}} \FunctionTok{log}\NormalTok{(X.T }\SpecialCharTok{+} \DecValTok{1}\NormalTok{)}
\NormalTok{PC\_log }\OtherTok{=} \FunctionTok{pcaCombo}\NormalTok{(}\FunctionTok{t}\NormalTok{(logX.T))}
\end{Highlighting}
\end{Shaded}

\pandocbounded{\includegraphics[keepaspectratio]{250613-Regional-PCA-Investigation_files/figure-latex/unnamed-chunk-1-2.pdf}}

\begin{Shaded}
\begin{Highlighting}[]
\DocumentationTok{\#\# IVT transformation}
\NormalTok{k }\OtherTok{=} \FloatTok{0.1}
\NormalTok{ivtX.T }\OtherTok{\textless{}{-}} \FunctionTok{IVT}\NormalTok{(X.T, k)}
\NormalTok{PC\_ivt }\OtherTok{=} \FunctionTok{pcaCombo}\NormalTok{(}\FunctionTok{t}\NormalTok{(ivtX.T))}
\end{Highlighting}
\end{Shaded}

\pandocbounded{\includegraphics[keepaspectratio]{250613-Regional-PCA-Investigation_files/figure-latex/unnamed-chunk-1-3.pdf}}

\begin{Shaded}
\begin{Highlighting}[]
\CommentTok{\# Marginal distribution before and after transformations}
\FunctionTok{par}\NormalTok{(}\AttributeTok{mfrow =} \FunctionTok{c}\NormalTok{(}\DecValTok{1}\NormalTok{,}\DecValTok{3}\NormalTok{))}
\FunctionTok{hist}\NormalTok{(X.T[}\DecValTok{1}\SpecialCharTok{:}\DecValTok{1000}\NormalTok{,], }\AttributeTok{main =} \StringTok{"Before any transformation"}\NormalTok{, }\AttributeTok{freq =}\NormalTok{ F)}
\FunctionTok{hist}\NormalTok{(logX.T[}\DecValTok{1}\SpecialCharTok{:}\DecValTok{1000}\NormalTok{,], }\AttributeTok{main =} \StringTok{"Log transformation"}\NormalTok{, }\AttributeTok{freq =}\NormalTok{ F)}
\FunctionTok{hist}\NormalTok{(ivtX.T[}\DecValTok{1}\SpecialCharTok{:}\DecValTok{1000}\NormalTok{,], }\AttributeTok{main =} \StringTok{"IVT transformation"}\NormalTok{,}\AttributeTok{freq =}\NormalTok{ F)}
\end{Highlighting}
\end{Shaded}

\pandocbounded{\includegraphics[keepaspectratio]{250613-Regional-PCA-Investigation_files/figure-latex/unnamed-chunk-1-4.pdf}}

\begin{Shaded}
\begin{Highlighting}[]
\CommentTok{\# Distribution of PC 1 values}
\FunctionTok{par}\NormalTok{(}\AttributeTok{mfrow =} \FunctionTok{c}\NormalTok{(}\DecValTok{1}\NormalTok{,}\DecValTok{3}\NormalTok{))}
\FunctionTok{hist}\NormalTok{(PC\_raw, }\AttributeTok{main =} \StringTok{"PC raw"}\NormalTok{, }\AttributeTok{freq =}\NormalTok{ F)}
\FunctionTok{hist}\NormalTok{(PC\_log, }\AttributeTok{main =} \StringTok{"PC log"}\NormalTok{, }\AttributeTok{freq =}\NormalTok{ F)}
\FunctionTok{hist}\NormalTok{(PC\_ivt, }\AttributeTok{main =} \StringTok{"PC ivt"}\NormalTok{, }\AttributeTok{freq =}\NormalTok{ F)}
\end{Highlighting}
\end{Shaded}

\pandocbounded{\includegraphics[keepaspectratio]{250613-Regional-PCA-Investigation_files/figure-latex/unnamed-chunk-1-5.pdf}}
\textbf{Observations}:

\begin{enumerate}
\def\labelenumi{\arabic{enumi}.}
\tightlist
\item
  The marginal distribution: after log(x+1), the distribution is still
  right skewed between 0 and 6, resembling exponential distributions.
  After IVT, the distribution is more symmetrically distributed between
  -2 and 4. Log and IVT were performed independently.
\item
  Classical PCA: before any transformation, all cancer types are
  collapsed together. After the log and IVT, the scatter plot spreads
  apart more, and the PC 1 plot becomes more symmetrically distributed
  around 0, and less spikes show up.
\item
  With log transformation, Blood Cancer and pancreatic cancer showed
  increasing separation from other cancer types. After IVT, only Blood
  Cancer has good separation from the others.
\end{enumerate}

\end{document}
